%%%% ijcai19.tex

\typeout{IJCAI-19 Instructions for Authors}

% These are the instructions for authors for IJCAI-19.

\documentclass{article}
\pdfpagewidth=8.5in
\pdfpageheight=11in
% The file ijcai19.sty is NOT the same than previous years'
\usepackage{ijcai19}

% Use the postscript times font!
\usepackage{times}
\usepackage{soul}
\usepackage{url}
\usepackage[hidelinks]{hyperref}
\usepackage[utf8]{inputenc}
\usepackage[small]{caption}
\usepackage{graphicx}
\usepackage{amsmath}
\usepackage{booktabs}
\usepackage{algorithm}
\usepackage{algorithmic}
\urlstyle{same}

% the following package is optional:
%\usepackage{latexsym}

% Following comment is from ijcai97-submit.tex:
% The preparation of these files was supported by Schlumberger Palo Alto
% Research, AT\&T Bell Laboratories, and Morgan Kaufmann Publishers.
% Shirley Jowell, of Morgan Kaufmann Publishers, and Peter F.
% Patel-Schneider, of AT\&T Bell Laboratories collaborated on their
% preparation.

% These instructions can be modified and used in other conferences as long
% as credit to the authors and supporting agencies is retained, this notice
% is not changed, and further modification or reuse is not restricted.
% Neither Shirley Jowell nor Peter F. Patel-Schneider can be listed as
% contacts for providing assistance without their prior permission.

% To use for other conferences, change references to files and the
% conference appropriate and use other authors, contacts, publishers, and
% organizations.
% Also change the deadline and address for returning papers and the length and
% page charge instructions.
% Put where the files are available in the appropriate places.

\title{IJCAI--19 Formatting Instructions}

% Single author syntax
\iffalse
\author{
    Sarit Kraus
    \affiliations
    Department of Computer Science, Bar-Ilan University, Israel \emails
    pcchair@ijcai19.org
}
\fi

% Multiple author syntax (remove the single-author syntax above and the \iffalse ... \fi here)
% Check the ijcai19-multiauthor.tex file for detailed instructions

\author{
Andre Diler$^1$
\and
Mehdi Chaid$^1$\and
Abderahmane Bouziane$^1$\
\affiliations
$^1$Département GIGL Polytechnique Montreal\\
\emails
andre.diler@polymtl.ca,
mehdi.chaid@polymtl.ca,
abderahmane.bouziane@polymtl.ca
}

\begin{document}

\maketitle

\begin{abstract}
  The {\it IJCAI--19 Proceedings} will be printed from electronic
  manuscripts submitted by the authors. The electronic manuscript will
  also be included in the online version of the proceedings. This paper
  provides the style instructions.
\end{abstract}

\section{Introduction}

https://hackernoon.com/a-brief-history-of-computer-vision-and-convolutional-neural-networks-8fe8aacc79f3

Modern Convolutional Neural Networks reach very high performances on complex computer vision tasks such as
Image classification, Image Segmentation \dots
These performances come from a long history of studies.

The first proof that edges are fundamental for vision came from a very influencal study of cognition:
  “Receptive fields of single neurons in the cat’s striate cortex”
described how biological neurons extracted features from images. It showed that certain types of
neurons were activated by edges.

The convolution of an image with a sobel filter is known to extract edges.
Other filters can extract other image features. However, the filters had to be handcrafted, and
the most significant features of an image had to be determined by a human.

Another paper showed that vision was hierachical. Low-level features (simple shapes like lines)
were combined to recognize high level features (like wheels, windows ...).

The famous Neocognition paper was the implementation of the idea of hierachical vision.
This multilayered neural network included convolutional layers with wheighted receptive fields (filters).
It was the first deep neural network.

The first modern convnet was introduce by
-  LeNet (http://yann.lecun.com/exdb/publis/pdf/lecun-01a.pdf).
It used backpropgation to automatically learn the filter values to extract meaningful features in images hierachically.


State of the art for Neural Networks
ResNet
Mask-RCNN etc

ConvNet were considered like a black box for a long time.
This is why efforts were made to analyze the internals of these networks.
One one hand, the shape hypothesis

One the other hand, the texture hypothesis.

Our objective is to test these hypothesis with diverse experiments.

To begin with, we reproduced the first experiment of the paper. This experiment
is represents the misconception we have about how CNNs learn.


We can see that the pretrained CNN recognized the elephant with just its texture.
More interestingly, it recognized it classified the cat shape with elephant texture as an elephant.
The texture hypothesis is more likely to be true.
Let's define more robust experiments to verify that.

\section{Methodology}


\section{Results}

\section{Discussion}

\subsubsection{References}

The references section is headed ``References'', printed in the same
style as a section heading but without a number. A sample list of
references is given at the end of these instructions. Use a consistent
format for references. The reference list should not include unpublished
work.

\subsection{Citations}

Citations within the text should include the author's last name and
the year of publication, for example~\cite{gottlob:nonmon}.  Append
lowercase letters to the year in cases of ambiguity.  Treat multiple
authors as in the following examples:~\cite{abelson-et-al:scheme}
or~\cite{bgf:Lixto} (for more than two authors) and
\cite{brachman-schmolze:kl-one} (for two authors).  If the author
portion of a citation is obvious, omit it, e.g.,
Nebel~\shortcite{nebel:jair-2000}.  Collapse multiple citations as
follows:~\cite{gls:hypertrees,levesque:functional-foundations}.
\nocite{abelson-et-al:scheme}
\nocite{bgf:Lixto}
\nocite{brachman-schmolze:kl-one}
\nocite{gottlob:nonmon}
\nocite{gls:hypertrees}
\nocite{levesque:functional-foundations}
\nocite{levesque:belief}
\nocite{nebel:jair-2000}

\section{Tables}

Tables are considered illustrations containing data. Therefore, they should also appear floated to the top (preferably) or bottom of the page, and with the captions below them.

\begin{table}
\centering
\begin{tabular}{lll}
\hline
Scenario  & $\delta$ & Runtime \\
\hline
Paris       & 0.1s  & 13.65ms     \\
Paris       & 0.2s  & 0.01ms      \\
New York    & 0.1s  & 92.50ms     \\
Singapore   & 0.1s  & 33.33ms     \\
Singapore   & 0.2s  & 23.01ms     \\
\hline
\end{tabular}
\caption{Latex default table}
\label{tab:plain}
\end{table}

\begin{table}
\centering
\begin{tabular}{lrr}
\toprule
Scenario  & $\delta$ (s) & Runtime (ms) \\
\midrule
Paris       & 0.1  & 13.65      \\
            & 0.2  & 0.01       \\
New York    & 0.1  & 92.50      \\
Singapore   & 0.1  & 33.33      \\
            & 0.2  & 23.01      \\
\bottomrule
\end{tabular}
\caption{Booktabs table}
\label{tab:booktabs}
\end{table}


\section*{Acknowledgments}

The preparation of these instructions and the \LaTeX{} and Bib\TeX{}
files that implement them was supported by Schlumberger Palo Alto
Research, AT\&T Bell Laboratories, and Morgan Kaufmann Publishers.
Preparation of the Microsoft Word file was supported by IJCAI.  An
early version of this document was created by Shirley Jowell and Peter
F. Patel-Schneider.  It was subsequently modified by Jennifer
Ballentine and Thomas Dean, Bernhard Nebel, Daniel Pagenstecher,
Kurt Steinkraus, Toby Walsh and Carles Sierra. The current version
has been prepared by Marc Pujol-Gonzalez and Francisco Cruz-Mencia.

\appendix

%% The file named.bst is a bibliography style file for BibTeX 0.99c
\bibliographystyle{named}
\bibliography{ijcai19}

\end{document}

